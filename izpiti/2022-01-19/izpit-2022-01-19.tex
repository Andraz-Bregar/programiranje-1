\documentclass[arhiv]{../izpit}
\usepackage{fouriernc}
\usepackage{xcolor}
\usepackage{fancyvrb}


\begin{document}
	
\izpit{Programiranje I: 1. izpit}{19.\ januar 2022}{
  Čas reševanja je 120 minut.
  Veliko uspeha!
}

%%%%%%%%%%%%%%%%%%%%%%%%%%%%%%%%%%%%%%%%%%%%%%%%%%%%%%%%%%%%%%%%%%%%%%%

\naloga

\emph{Nalogo rešujte v OCamlu.}

\podnaloga
  Napišite predikat \verb|sta_pravokotna : int * int * int -> int * int * int -> bool|, ki pove, ali sta dva vektorja pravokotna. Vektorje predstavimo s trojicami celih števil.

\podnaloga
  Napišite funkcijo \verb|postkompozicija : ('a -> 'b) -> ('b -> 'c) -> 'a -> 'c|, ki sprejme dve funkciji in vrne njun kompozitum.
  \begin{verbatim}
    # let deli_z m n = n / m;;
    # postkompozicija ((+) 82) (deli_z 82) 3362;;
    - : int = 42
    # postkompozicija (deli_z 82) ((+) 82) 3362;;
    - : int = 123
  \end{verbatim}

\podnaloga
  Napišite funkcijo \verb|dopolni : 'a -> 'a option list -> 'a list|, ki sprejme privzeto vrednost in seznam morebitnih vrednosti ter vrne seznam, v katerem namesto manjkajočih vrednosti nastopa privzeta, obstoječe vrednosti pa ostanejo nespremenjene. Za vse točke naj bo funkcija repno rekurzivna.
  \begin{verbatim}
    # dopolni 3 [Some 1; None; Some 2];;
    - : int list = [1; 3; 2]
  \end{verbatim}

\podnaloga
  Naravno število v drugačni bazi predstavimo s seznamom celih števil, kjer so na zadnjem mestu enice, na predzadnjem ``desetice'' in tako naprej.
  
  Napišite funkcijo \verb|pretvori : int -> int list -> int|, ki sprejme seznam pozitivnih celih števil in bazo ter vrne podano število v desetiškem sistemu kot celo število.
  \begin{verbatim}
    # pretvori 8 [6; 4; 4; 2];; (* 6*8^3 + 4*8^2 + 4*8 + 2 *)
    - : int = 3362
  \end{verbatim}

%%%%%%%%%%%%%%%%%%%%%%%%%%%%%%%%%%%%%%%%%%%%%%%%%%%%%%%%%%%%%%%%%%%%%%%

\naloga

\emph{Nalogo rešujte v OCamlu.}

\noindent
V datotečnih sistemih, sistemih za nadzor različic in kriptovalutah za učinkovito zagotavljanje pristnosti podatkov uporabljamo \emph{Merkleova drevesa}. To so običajna dvojiška drevesa, le da vsako vozlišče poleg podatka in otrok hrani še svojo \emph{zgostitev} (angleško \emph{hash}), ki je celo število, ki tvori nekakšen podpis celotne vsebine drevesa.

\begin{verbatim}
  type 'a merkle = List | Vozlisce of 'a vozlisce
  and 'a vozlisce = {
    levo : 'a merkle;
    podatek : 'a;
    desno : 'a merkle;
    zgostitev : int;
  }
\end{verbatim}

Zgostitev drevesa izračunamo z zgostitvenimi funkcijami, ki sprejmejo zgostitev levega otroka, podatek v vozlišču in zgostitev desnega otroka:
\begin{verbatim}
  type 'a zgostitev = int -> 'a -> int -> int
\end{verbatim}
Za zgostitev praznega drevesa izberemo število 0.

Na primer, za zgostitveno funkcijo:
\begin{verbatim}
  let primer_h l p d = ((l * 3) + (p * 5) + (d * 7)) mod 11
\end{verbatim}
je spodnje drevo veljavno Merkleovo drevo:
\begin{verbatim}
  let drevo : int merkle = Vozlisce {
    levo = Vozlisce {
      levo = Vozlisce { levo = List; podatek = 10; desno = List; zgostitev = 6 };
      podatek = 14;
      desno = Vozlisce { levo = List; podatek = 474; desno = List; zgostitev = 5 };
      zgostitev = 2;
    };
    podatek = 57;
    desno = Vozlisce {
      levo = List;
      podatek = 12;
      desno = Vozlisce { levo = List; podatek = 513; desno = List; zgostitev = 2 };
      zgostitev = 8;
    };
    zgostitev = 6;
  }
\end{verbatim}

\podnaloga
  Napišite predikat \verb|preveri : 'a zgostitev -> 'a merkle -> bool|, ki sprejme zgostitveno funkcijo in drevo ter preveri, ali so vse zgostitve v drevesu pravilno izračunane.

\podnaloga
  Napišite funkcijo \verb|prestej_napacne : 'a zgostitev -> 'a merkle -> int|, ki vzame zgostitveno funkcijo in drevo ter vrne število napačno poračunanih zgostitev v drevesu. Zgostitev vozliča je napačna, če ne ustreza definiciji računanja zgostitve za vozlišče, ne glede na to, ali sta zgostitvi otrok pravilni ali ne.

\podnaloga
  Napišite funkcijo \verb|popravi : 'a zgostitev -> 'a merkle -> 'a merkle|, ki sprejme (potencialno napačno) drevo in vrne drevo iste oblike z istimi podatki, le da so vse zgostitve v njem pravilno izračunane.

%%%%%%%%%%%%%%%%%%%%%%%%%%%%%%%%%%%%%%%%%%%%%%%%%%%%%%%%%%%%%%%%%%%%%%%

\naloga

\emph{Nalogo lahko rešujete v Pythonu ali OCamlu.}

\noindent
Radko Razkurnik je poklicni internetni trol, ki ljudi s pogostim menjanjem stališč zapenja v nekonstruktivne debate. Glede aktualnega družbenega dogajanja zna zagovarjati $n$ različnih stališč in vsakič, ko preklopi iz stališča $i$ v $j$, s svojo nedoslednostjo spodbudi $k_{ij}$ komentarjev. (Seveda nekaj komentarjev sproži tudi z vztrajanjem pri stališču $i$). Dan je kratek, zato lahko Radko napiše največ $m$ komentarjev, pri čemer je $m$ precej večji od $n$, saj Radko hitreje piše kot misli. Kako naj izbere svoja stališča $s_1, s_2, \dots, s_m$, da bo skupaj sprožil največ komentarjev, kar jih lahko?

Na primer pri matriki
\[
  k = \begin{pmatrix}
    5 & 25 & 3 & 2 \\
    1 & 4 & 25 & 1 \\
    2 & 1 & 3 & 2 \\
    4 & 1 & 40 & 5
  \end{pmatrix}
\]
so optimalne vsote pri $m = 1, 2, 4, 6, 9$ enake:
\begin{align*}
  k_{43} &= 40 \\
  k_{12} + k_{23} &= 50 \\
  k_{12} + k_{23} + k_{34} + k_{43} &= 92 \\
  k_{12} + k_{23} + k_{34} + k_{43} + k_{34} + k_{43} &= 134 \\
  k_{43} + k_{34} + k_{43} + k_{34} + k_{43} + k_{34} + k_{43} + k_{34} + k_{43} &= 208
\end{align*}
Napišite \emph{čim bolj učinkovito} funkcijo \verb|trololo|, ki sprejme matriko $(k_{ij})_{1 \le i, j \le n} \in \mathbb{N}^{n \times n}$ ter $m \in \mathbb{N}$ in izračuna zgornji seznam.

\end{document}
